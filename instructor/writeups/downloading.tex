\documentclass[12pt]{article}

\usepackage{times}
\usepackage{fullpage}
\usepackage{alltt}
\newenvironment{code}{\begin{alltt}\small}{\end{alltt}}

\usepackage{color}
\usepackage{hyperref}
%% Colored hyperlink 
\newcommand{\cref}[2]{\href{#1}{\color{blue}#2}}
%% Colored hyperlink showing link in TT font
% \newcommand{\chref}[1]{\href{#1}{\small\tt \color{blue}#1}}
\newcommand{\hcref}[1]{\cref{#1}{\small\tt #1}}

\begin{document}

\begin{center}
{\LARGE\bf Summer School on Formal Techniques\\ [1.5 ex]
Boolean (Un)Satisfiability \\[2ex]
Obtaining Class Materials}\\ [2 ex]
June, 2022
\end{center}

The labs consist of three C/C++ programs, several Python programs, PDF
documents, and some data files.

\section*{Available Material}

Lectures:
\begin{itemize}
\item \cref{https://github.com/rebryant/unsat-tutorial/blob/main/instructor/slides/lecture-1a-intro.pdf}{Lecture 1A: (Un)Satisfiability}
\item \cref{https://github.com/rebryant/unsat-tutorial/blob/main/instructor/slides/lecture-1b-proof.pdf}{Lecture 1b: Unsatisfiability Proofs}
\item \cref{https://github.com/rebryant/unsat-tutorial/blob/main/instructor/slides/lecture-2a-bdd-intro.pdf}{Lecture 2a: Introduction to BDDs}
\item \cref{https://github.com/rebryant/unsat-tutorial/blob/main/instructor/slides/lecture-2b-bdd-proof.pdf}{Lecture 2a: Proof Generation with BDDs}
\end{itemize}  

Labs:
\begin{itemize}
\item \cref{https://github.com/rebryant/unsat-tutorial/blob/main/instructor/writeups/downloading.pdf}{Obtaining Lab Materials}
\item \cref{https://github.com/rebryant/unsat-tutorial/blob/main/instructor/writeups/lab1.pdf}{Lab 1}
\item \cref{https://github.com/rebryant/unsat-tutorial/blob/main/instructor/writeups/lab2.pdf}{Lab 2}
\item \cref{https://github.com/rebryant/unsat-tutorial/blob/main/instructor/writeups/lab1-solution.pdf}{Lab 1 Solution}
\item \cref{https://github.com/rebryant/unsat-tutorial/blob/main/instructor/writeups/lab2-solution.pdf}{Lab 2 Solution}
\end{itemize}
  

\newpage

\section*{Setting up Labs}

\subsection*{Via Github}

You must have the {\tt git} progam installed on your machine.

\begin{code}
git clone https://github.com/rebryant/unsat-tutorial
cd unsat-tutorial
make install
\end{code}

\subsection*{Via Dockerhub}

You must have the {\tt docker} program installed on your machine.

\begin{code}
  docker pull randalbryant/ssft22
  docker run -it randalbryant/ssft22 bash
  login ssft22 \textit{(Password = ssft22)}
  cd unsat-tutorial
\end{code}

\subsection*{Saving your Docker Container}


The above \texttt{docker run} command creates and runs a {\em
  container} based on the specified Docker {\em image}.  If you exit the shell
on your docker container,\footnote{Actually, you need to exit twice---once from your user login and once from the shell.}
any changes you have made to any files
would normally be lost.  Instead you must \texttt{commit} your
container, but you can't commit back to the version on Dockerhub.

Instead, you should commit to a local container.  You should do this
before you exit the shell.  Here's the procedure:

\begin{enumerate}
\item In a separate terminal window run the command:
  \begin{code}
docker ps
  \end{code}
  to see the list of running containers.  It will look something like the following:
\begin{code}
CONTAINER ID   IMAGE                 COMMAND   CREATED
16640d5ce162   randalbryant/ssft22   "bash"    8 minutes ago
5836b1b74ccf   gcc:latest            "bash"    6 hours ago 
\end{code}

\item Perform a commit using the hexadecimal Container Id:
\begin{code}
docker container commit 16640d5ce162 ssft22_local
\end{code}
You now exit the container shell.

\item You can later pick up where you left off at by executing
\begin{code}
docker container commit 16640d5ce162 ssft22_local
\end{code}
and login again.
\end{enumerate}

\end{document}
