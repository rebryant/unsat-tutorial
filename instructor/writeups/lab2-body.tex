
\usepackage[pdftex]{graphicx}


\usepackage{exercise}
\usepackage{enumitem}
\usepackage{color}
\usepackage{booktabs}
\usepackage{tikz}

\newcommand{\proofhu}[1]{${\it HU}(\texttt{#1})$}
\newcommand{\prooflu}[1]{${\it LU}(\texttt{#1})$}
\newcommand{\proofhd}[1]{${\it HD}(\texttt{#1})$}
\newcommand{\proofld}[1]{${\it LD}(\texttt{#1})$}

\newcommand{\pathname}[2]{\textit{ROOT}\texttt{/}\texttt{#1}\texttt{/}\texttt{#2}}

\begin{document}
\begin{center}
{\LARGE\bf Summer School on Formal Techniques\\ [1.5 ex]
Boolean (Un)Satisfiability \\[2ex]
\showexamname{2}}\\ [2 ex]
June, 2022
\end{center}
\section*{Explanation}

These exercises are designed to provide a deeper understanding of the
operation of Boolean satisfiability (SAT) solvers, especially when
applied to unsatisfiable formulas.  A key requirement is that solver
be able to generate a proof of unsatisfiability in such cases.

The provided problems range in how much time and effort is required,
and whether any programming is involved.  Each problem has an
associated {\em level}, according to the following standard:
\begin{description}
\item[I:] Simple pencil-and-paper exercises designed to provide a
  concrete examples for the concepts presented.  Doing these will
  help you gain confidence in the concepts being presented
\item[II:] More challenging pencil-and-paper exercises, or algorithmic
  and experimental activities.  These may running solvers on some
  benchmarks.
\item[III:] Deeper explorations.  These may require devising new
  algorithms, writing code, and performing experiments that go beyond
  the core lecture material.
\end{description}

All file names are specified in this document are given as path names
of the form
\pathname{\textit{DIR}}{\textit{FILE}}
where
\textit{ROOT} indicates the root of the directory structure,
\textit{DIR} is either ``\texttt{files}'' or ``\texttt{generators},''
and \textit{FILE} is the file name.


\newpage

\section*{Using the Provided Programs}

Here is how to use the provided tools.  In the following, we assume
{\it FILE} is the common base name for a set of files having different extensions.


\subsection*{TBSAT}

\begin{itemize}
\item Running in direct mode without proof generation, and generating up to $K$ solutions
\item[]\pathname{tbuddy/src/tbsat}{tbsat} \texttt{-i} {\it FILE}\texttt{.cnf} \texttt{-m} $K$
\item Running in direct mode with proof generation
\item[]\pathname{tbuddy/src/tbsat}{tbsat} \texttt{-i} {\it FILE}\texttt{.cnf} \texttt{-o} {\it FILE}\texttt{.lrat}
\item Running in bucket mode with proof generation
\item[]\pathname{tbuddy/src/tbsat}{tbsat} \texttt{-b} \texttt{-i} {\it FILE}\texttt{.cnf} \texttt{-o} {\it FILE}\texttt{.lrat}
\item Running with a schedule file and proof generation
\item[]\pathname{tbuddy/src/tbsat}{tbsat} \texttt{-s} {\it FILE}\texttt{.schedule} \texttt{-i} {\it FILE}\texttt{.cnf} \texttt{-o} {\it FILE}\texttt{.lrat}
\end{itemize}

\subsection*{LRAT-CHECK}

\begin{itemize}
\item Checking proof
\item[]\pathname{drat-trim}{lrat-check} {\it FILE}\texttt{.cnf} {\it FILE}\texttt{.lrat}
\end{itemize}


\newpage

\section*{BDD-Based SAT Solving}

Consider the following unsatisfiable set of parity constraints
\begin{displaymath}
\begin{array}{ccccccc}
a & \oplus & b  &        &   & = & 1 \\
  &        & b  & \oplus & c & = & 1 \\
a & \oplus &    &        & c & = & 1 \\
\end{array}
\end{displaymath}

We refer to this formula as ${\it CYC}(3)$, where ${\it CYC}(n)$
represents a cyclic chain of odd parity constraints for $n$ variables.
It's easy to see that ${\it CYC}(n)$ is unsatisfiable for odd $n$.

File
\texttt{cyc3.cnf} encodes these constraints in clausal form
numbering variables $a$, $b$, and $c$ as 1 through 3:
\begin{lstlisting}
p cnf 3 6
 1  2 0
-1 -2 0
 2  3 0
-2 -3 0
 1  3 0
-1 -3 0
\end{lstlisting}

\newpage

\begin{problem}{I}
When running TBSAT, it can be directed to operate in ``verbose'' mode, adding comments to the generated proof file.
Here is an excerpt of the comments generated when running in direct mode on \texttt{cyc3.cnf} (See Slide~\#14 in Lecture~2a).
The full proof is in file \texttt{cyc3-dct.lrat}.
\begin{lstlisting}
c Input Clause #1: 2 1 0
c Input Clause #2: -2 -1 0
c Input Clause #3: 3 2 0
c Input Clause #4: -3 -2 0
c Input Clause #5: 3 1 0
c Input Clause #6: -3 -1 0
c Defining clauses for node N4 = ITE(V1 (level=1), N1, N0)
c Defining clauses for node N5 = ITE(V1 (level=1), N0, N1)
c Defining clauses for node N6 = ITE(V2 (level=2), N1, N0)
c Defining clauses for node N7 = ITE(V2 (level=2), N0, N1)
c Defining clauses for node N8 = ITE(V3 (level=3), N1, N0)
c Defining clauses for node N9 = ITE(V3 (level=3), N0, N1)
c Defining clauses for node N10 = ITE(V1 (level=1), N1, N6)
c Validate BDD representation of Clause #1.  Node = N10.
c Defining clauses for node N11 = ITE(V1 (level=1), N7, N1)
c Validate BDD representation of Clause #2.  Node = N11.
c Defining clauses for node N12 = ITE(V2 (level=2), N1, N8)
c Validate BDD representation of Clause #3.  Node = N12.
c Defining clauses for node N13 = ITE(V2 (level=2), N9, N1)
c Validate BDD representation of Clause #4.  Node = N13.
c Defining clauses for node N14 = ITE(V1 (level=1), N1, N8)
c Validate BDD representation of Clause #5.  Node = N14.
c Defining clauses for node N15 = ITE(V1 (level=1), N9, N1)
c Validate BDD representation of Clause #6.  Node = N15.
c Defining clauses for node N16 = ITE(V1 (level=1), N7, N6)
c Generating proof that N10 & N11 --> N16
c Defining clauses for node N17 = ITE(V2 (level=2), N9, N8)
c Generating proof that N12 & N13 --> N17
c Defining clauses for node N18 = ITE(V1 (level=1), N9, N8)
c Generating proof that N14 & N15 --> N18
c Defining clauses for node N19 = ITE(V2 (level=2), N9, N0)
c Generating proof that N6 & N17 --> N19
c Defining clauses for node N20 = ITE(V2 (level=2), N0, N8)
c Generating proof that N7 & N17 --> N20
c Defining clauses for node N21 = ITE(V1 (level=1), N20, N19)
c Generating proof that N16 & N17 --> N21
c Validate empty clause for node N0 = N18 & N21
\end{lstlisting}

From these comments it is possible to trace how the program converted
the clauses into BDDs, formed their conjunctions, and detected that
this yielded BDD leaf $\bot$.  Understanding these steps will help you
better understand how the program operates.

Here are some guidelines on the notation:
\begin{itemize}
\item Nodes \texttt{N1} and \texttt{N0} denote the two leaf nodes, with values 1 and 0, respectively
\item Nonterminal BDD nodes are given names of the form \texttt{N}$z$, where $z$ is the integer extension variable associated with the node.
\item Nonterminal nodes are written
as \texttt{ITE(V$x$, N$h$, N$l$)}, where $x$ indicates the variable,
and \texttt{N}$h$ and \texttt{N}$l$ indicate the two children.
\end{itemize}

\begin{choice}
\item
The following nodes each correspond to conjunctions of some set of the
input clauses.  Fill in the following table with that information, as
is illustrated with the first entry.
\begin{center}
\renewcommand{\arraystretch}{1.1}
\begin{tabular}{cl}
\toprule
\makebox[.5in]{Node} & \makebox[1.5in]{Clauses} \\
\midrule
\texttt{N16} & \texttt{1}, \texttt{2} \\
\texttt{N17} & \squick{\texttt{3}, \texttt{4}} \\
\texttt{N18} & \squick{\texttt{5}, \texttt{6}} \\
\texttt{N21} & \squick{\texttt{1}, \texttt{2}, \texttt{3}, \texttt{4}} \\
\bottomrule
\end{tabular}
\end{center}

\item
Draw the BDD having node \texttt{N21} as root.
\solution{1.5in}{\begin{center}\input{dd/n21.tex}\end{center}}
\item
Looking at the paths from the root to Leaf 1 in this BDD, what
constraint do these place on the relation between  $a$ and $b$?
\solution{1in}{The only two paths require that $a=b$.}
\item
What does that constraint imply when the conjunction of the BDDs with roots \texttt{N21} and \texttt{N18} is formed?
\solution{1in}{
The BDD with root \texttt{N18} is a representation of $a \oplus b$,
and so it requires $a \not = b$.  Their conjunction is
therefore $\bot$.
}
\end{choice}
\end{problem}

\newpage

\begin{problem}{I}
When running TBSAT on \texttt{cyc3.cnf} in bucket mode (see Slide~\#16--17 of Lecture 2b), the generated proof contains these comments.
The full proof is in file \texttt{cyc3-bkt.lrat}.

\begin{lstlisting}
c Input Clause #1: 2 1 0
c Input Clause #2: -2 -1 0
c Input Clause #3: 3 2 0
c Input Clause #4: -3 -2 0
c Input Clause #5: 3 1 0
c Input Clause #6: -3 -1 0
c Defining clauses for node N4 = ITE(V1 (level=1), N1, N0)
c Defining clauses for node N5 = ITE(V1 (level=1), N0, N1)
c Defining clauses for node N6 = ITE(V2 (level=2), N1, N0)
c Defining clauses for node N7 = ITE(V2 (level=2), N0, N1)
c Defining clauses for node N8 = ITE(V3 (level=3), N1, N0)
c Defining clauses for node N9 = ITE(V3 (level=3), N0, N1)
c Defining clauses for node N10 = ITE(V1 (level=1), N1, N6)
c Validate BDD representation of Clause #1.  Node = N10.
c Defining clauses for node N11 = ITE(V1 (level=1), N7, N1)
c Validate BDD representation of Clause #2.  Node = N11.
c Defining clauses for node N12 = ITE(V2 (level=2), N1, N8)
c Validate BDD representation of Clause #3.  Node = N12.
c Defining clauses for node N13 = ITE(V2 (level=2), N9, N1)
c Validate BDD representation of Clause #4.  Node = N13.
c Defining clauses for node N14 = ITE(V1 (level=1), N1, N8)
c Validate BDD representation of Clause #5.  Node = N14.
c Defining clauses for node N15 = ITE(V1 (level=1), N9, N1)
c Validate BDD representation of Clause #6.  Node = N15.
c Defining clauses for node N16 = ITE(V1 (level=1), N7, N6)
c Generating proof that N10 & N11 --> N16
c Defining clauses for node N17 = ITE(V1 (level=1), N9, N8)
c Generating proof that N14 & N15 --> N17
c Defining clauses for node N18 = ITE(V2 (level=2), N8, N0)
c Defining clauses for node N19 = ITE(V2 (level=2), N0, N9)
c Defining clauses for node N20 = ITE(V1 (level=1), N19, N18)
c Generating proof that N16 & N17 --> N20
c Defining clauses for node N21 = ITE(V2 (level=2), N8, N9)
c Generating proof that N20 --> N21
c Defining clauses for node N22 = ITE(V2 (level=2), N9, N8)
c Generating proof that N12 & N13 --> N22
c Validate empty clause for node N0 = N21 & N22
\end{lstlisting}

\newpage
\begin{choice}
\item Which clauses are conjuncted to form the BDD with root node \texttt{N20}?  What does this BDD represent?
\solution{1in}{Clauses \texttt{1}, \texttt{2}, \texttt{4}, and \texttt{5}.
These are the clauses containing the top variable $a$. This BDD is a conjunction of the BDDs for Bucket A\@.}

\item Draw a diagram of the BDD with root node \texttt{N20}.
\solution{1.5in}{\begin{center}\input{dd/n20.tex}\end{center}}

\item The BDD with root node \texttt{N21} is the result of existentially quantifying $a$ from \texttt{N20}, and
the BDD with root node \texttt{N22} is the result of conjuncting clauses $C_3$ and $C_4$.  
Draw a diagram showing both of these BDDs together.
\solution{1.5in}{\begin{center}\input{dd/n21-22.tex}\end{center}}

\item What happens when the conjunction of the BDDs with root nodes \texttt{N21} and \texttt{N22} is formed?  Explain
\solution{1.0in}{\texttt{N21} requires $b$ and $c$ to have even parity, while \texttt{N22} requires them to have odd parity.
Their conjunction is therefore $\bot$.}

\end{choice}
\end{problem}

\newpage
\section*{Proof Generation with BDDs}

The following shows a portion of the LRAT proof file generated when TBSAT is applied to \texttt{cyc3.cnf}.
The full proof is in file \texttt{cyc3-dct.lrat}.
\begin{lstlisting}
c Input Clause #1: 2 1 0

c Defining clauses for node N6 = ITE(V2 (level=2), N1, N0)
15 6 -2 0  0
18 -6 2 0 -15 0

c Defining clauses for node N10 = ITE(V1 (level=1), N1, N6)
31 10 -1 0  0
32 10 -6 1 0  0
34 -10 6 1 0 -31 -32 0
c Validate BDD representation of Clause #1.  Node = N10.
35 10 0 31 32 15 1 0
\end{lstlisting}

\begin{problem}{I}
As the comments indicate, clauses \texttt{15}, \texttt{18}, and \texttt{31}--\texttt{34} are defining
clauses for two of the nodes.  With LRAT, extension variables are
introduced via such clauses, where the literal for the extension variable is listed
first.  Rather than the antecedents that we saw in the RUP clause
examples of the lectures, the second portion of a defining clause
lists any previous clauses containing the opposite literal of that of
the extension variable, but these are also negated.  So for example,
Clause \texttt{18} lists \texttt{-15}, while clause \texttt{34}
lists \texttt{-31} and \texttt{-32}.  (Understanding this part of the
syntax is not critical here.)

\begin{choice}
\item In general, there can be up to four defining clauses per BDD node, as was shown on Slide~\#4 of Lecture~2b.
But, when a clause degenerates to a tautology, it is not included in
the proof.
Show how clauses \texttt{15} and \texttt{16} match up to the defining clauses for node \texttt{N6}.
\solution{0.5in}{
Clause \texttt{15} represents \proofhu{N6}, and
clause \texttt{18} represents \proofld{N6}.
Clauses \prooflu{N6} and \proofhd{N6} are tautologies.
}

\item
Show how clauses \texttt{31}, \texttt{32}, and \texttt{34} match up to the defining clauses for node \texttt{N10}.
\solution{0.5in}{
Clause \texttt{31} represents \proofhu{N10},
clause \texttt{32} represents \prooflu{N10},
and clause \texttt{34} represents \proofld{N10}.
Clause \proofhd{N10} is a tautology.
}

\item
Simulate the RUP proof steps to provide a justification of
unit clause \texttt{35}, indicating that node \texttt{N10} is the TBDD representation of input clause \texttt{1}.
\solution{0.5in}{
Starting with literal \texttt{-10},
clause \texttt{31} generates unit \texttt{-1}, and \texttt{32}
generates unit \texttt{-6}.  Clause \texttt{15} then generates
unit \texttt{-2}, while clause \texttt{1} generates a conflict.
}
\end{choice}
\end{problem}

\newpage
The following shows a portion of the LRAT proof file generated when TBSAT is applied to \texttt{cyc3.cnf} in direct mode.
The full proof is in file \texttt{cyc3-dct.lrat}.
\begin{lstlisting}
c Input Clause #1: 2 1 0
c Input Clause #2: -2 -1 0

c Defining clauses for node N6 = ITE(V2 (level=2), N1, N0)
15 6 -2 0  0
18 -6 2 0 -15 0
c Defining clauses for node N7 = ITE(V2 (level=2), N0, N1)
20 7 2 0  0
21 -7 -2 0 -20 0

c Defining clauses for node N10 = ITE(V1 (level=1), N1, N6)
31 10 -1 0  0
32 10 -6 1 0  0
34 -10 6 1 0 -31 -32 0
c Validate BDD representation of Clause #1.  Node = N10.
35 10 0 31 32 15 1 0

c Defining clauses for node N11 = ITE(V1 (level=1), N7, N1)
36 11 -7 -1 0  0
37 11 1 0  0
38 -11 7 -1 0 -36 -37 0
c Validate BDD representation of Clause #2.  Node = N11.
40 11 0 37 36 20 2 0

c Defining clauses for node N16 = ITE(V1 (level=1), N7, N6)
61 16 -7 -1 0  0
62 16 -6 1 0  0
63 -16 7 -1 0 -61 -62 0
64 -16 6 1 0 -61 -62 0
c Generating proof that N10 & N11 --> N16
65 16 -11 -10 -1 0 61 38 0
66 16 -11 -10 0 65 62 34 0

c Validate unit clause for node N16 = N10 & N11
67 16 0 35 40 66 0
\end{lstlisting}

\newpage

\begin{choice}
\item
Proof clauses \texttt{65} and \texttt{66} justify that
node \texttt{N16} is the conjunction of BDDs representing input
clauses \texttt{1} and \texttt{2}.  It is a special case of the
general form shown on Slide~\#7 of Lecture~2b, where the
justifications for the two recursive calls are the tautologies
$a \land 1 \rightarrow a$ and $\overline{a} \land 1 \rightarrow \overline{a}$.
Match up the other clauses to those shown on Slide \#7.

\solution{1in}{
Clause \texttt{61} represents \proofhu{N16}.
Clause \texttt{38} represents \proofhd{N11}.
Clause \texttt{62} represents \prooflu{N16}.
Clause \texttt{34} represents \proofld{N20}.
}

\item 
Simulate the RUP proof steps to justify proof clauses \texttt{65} and \texttt{66}.

\solution{1in}{
The proof for clause \texttt{65} starts with literals \texttt{-16}, \texttt{11}, \texttt{10}, and \texttt{1}. Clause \texttt{61} gives \texttt{-7}.
Clause Clause \texttt{38} gives conflict.

The proof for clause \texttt{66} starts with literals \texttt{-16},
\texttt{11}, and
\texttt{10}.
Clause
\texttt{65}
gives
\texttt{-1}.
Clause
\texttt{62}
gives
\texttt{-6}.
Clause
\texttt{34}
gives conflict.
}

\item
What is the significance of proof clause \texttt{67}?  What is its justification?
\solution{1in}{
This step justifies the unit clause for \texttt{N16}, showing that \texttt{N16} is the TBDD representation for the conjunction of input clauses \texttt{1} and \texttt{2}.  It follows from 1) the unit clauses for the BDDs representing the two clauses and 2) the proof for the ApplyAnd operation on step \texttt{66}.
}

\end{choice}

\end{document}


