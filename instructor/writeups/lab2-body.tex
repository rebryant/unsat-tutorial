
\usepackage[pdftex]{graphicx}


\usepackage{exercise}
\usepackage{enumitem}
\usepackage{color}
\usepackage{booktabs}
\usepackage{tikz}
\usepackage{bm}

\usepackage{hyperref}
%% Colored hyperlink 
\newcommand{\cref}[2]{\href{#1}{\color{blue}#2}}
%% Colored hyperlink showing link in TT font
% \newcommand{\chref}[1]{\href{#1}{\small\tt \color{blue}#1}}
\newcommand{\hcref}[1]{\cref{#1}{\small\tt #1}}


\newcommand{\proofhu}[1]{${\it HU}(\texttt{#1})$}
\newcommand{\prooflu}[1]{${\it LU}(\texttt{#1})$}
\newcommand{\proofhd}[1]{${\it HD}(\texttt{#1})$}
\newcommand{\proofld}[1]{${\it LD}(\texttt{#1})$}

\begin{document}
\begin{center}
{\LARGE\bf Summer School on Formal Techniques\\ [1.5 ex]
Boolean (Un)Satisfiability \\[2ex]
\showexamname{2}}\\ [2 ex]
June, 2022
\end{center}
\section*{Explanation}

These exercises are designed to provide a deeper understanding of the
operation of Boolean satisfiability (SAT) solvers, especially when
applied to unsatisfiable formulas.  A key requirement is that solver
be able to generate a proof of unsatisfiability in such cases.

The provided problems range in how much time and effort is required,
and whether any programming is involved.  Each problem has an
associated {\em level}, according to the following standard:
\begin{description}
\item[I:] Simple pencil-and-paper exercises designed to provide a
  concrete examples for the concepts presented.  Doing these will
  help you gain confidence in the concepts being presented
\item[II:] More challenging pencil-and-paper exercises, or algorithmic
  and experimental activities.  These may running solvers on some
  benchmarks.
\item[III:] Deeper explorations.  These may require devising new
  algorithms, writing code, and performing experiments that go beyond
  the core lecture material.
\end{description}


\newpage

\section*{BDD-Based SAT Solving}

Consider the following unsatisfiable set of parity constraints
\begin{displaymath}
\begin{array}{ccccccc}
a & \oplus & b  &        &   & = & 1 \\
  &        & b  & \oplus & c & = & 1 \\
a & \oplus &    &        & c & = & 1 \\
\end{array}
\end{displaymath}

We refer to this formula as ${\it CYC}(3)$, where ${\it CYC}(n)$
represents a cyclic chain of odd parity constraints for $n$ variables.
It's easy to see that ${\it CYC}(n)$ is unsatisfiable for odd $n$.

File
\texttt{cyc3.cnf} encodes these constraints in clausal form
numbering variables $a$, $b$, and $c$ as 1 through 3:
\begin{lstlisting}
p cnf 3 6
 1  2 0
-1 -2 0
 2  3 0
-2 -3 0
 1  3 0
-1 -3 0
\end{lstlisting}

\newpage

\begin{problem}{I}
When running TBSAT, it can be directed to operate in ``verbose'' mode, adding comments to the generated proof file.
Here is an excerpt of the comments generated when running in direct mode on \texttt{cyc3.cnf} (See Slide~\#14 in Lecture~2a).
The full proof is in file \texttt{files/cyc3-dct.lrat}.
\begin{lstlisting}
c Input Clause #1: 2 1 0
c Input Clause #2: -2 -1 0
c Input Clause #3: 3 2 0
c Input Clause #4: -3 -2 0
c Input Clause #5: 3 1 0
c Input Clause #6: -3 -1 0
c Defining clauses for node N4 = ITE(V1 (level=1), N1, N0)
c Defining clauses for node N5 = ITE(V1 (level=1), N0, N1)
c Defining clauses for node N6 = ITE(V2 (level=2), N1, N0)
c Defining clauses for node N7 = ITE(V2 (level=2), N0, N1)
c Defining clauses for node N8 = ITE(V3 (level=3), N1, N0)
c Defining clauses for node N9 = ITE(V3 (level=3), N0, N1)
c Defining clauses for node N10 = ITE(V1 (level=1), N1, N6)
c Validate BDD representation of Clause #1.  Node = N10.
c Defining clauses for node N11 = ITE(V1 (level=1), N7, N1)
c Validate BDD representation of Clause #2.  Node = N11.
c Defining clauses for node N12 = ITE(V2 (level=2), N1, N8)
c Validate BDD representation of Clause #3.  Node = N12.
c Defining clauses for node N13 = ITE(V2 (level=2), N9, N1)
c Validate BDD representation of Clause #4.  Node = N13.
c Defining clauses for node N14 = ITE(V1 (level=1), N1, N8)
c Validate BDD representation of Clause #5.  Node = N14.
c Defining clauses for node N15 = ITE(V1 (level=1), N9, N1)
c Validate BDD representation of Clause #6.  Node = N15.
c Defining clauses for node N16 = ITE(V1 (level=1), N7, N6)
c Generating proof that N10 & N11 --> N16
c Defining clauses for node N17 = ITE(V2 (level=2), N9, N8)
c Generating proof that N12 & N13 --> N17
c Defining clauses for node N18 = ITE(V1 (level=1), N9, N8)
c Generating proof that N14 & N15 --> N18
c Defining clauses for node N19 = ITE(V2 (level=2), N9, N0)
c Generating proof that N6 & N17 --> N19
c Defining clauses for node N20 = ITE(V2 (level=2), N0, N8)
c Generating proof that N7 & N17 --> N20
c Defining clauses for node N21 = ITE(V1 (level=1), N20, N19)
c Generating proof that N16 & N17 --> N21
c Validate empty clause for node N0 = N18 & N21
\end{lstlisting}

From these comments it is possible to trace how the program converted
the clauses into BDDs, formed their conjunctions, and detected that
this yielded BDD leaf $\bot$.  Understanding these steps will help you
better understand how the program operates.

Here are some guidelines on the notation:
\begin{itemize}
\item Nodes \texttt{N1} and \texttt{N0} denote the two leaf nodes, with values 1 and 0, respectively
\item Nonterminal BDD nodes are given names of the form \texttt{N}$z$, where $z$ is the integer extension variable associated with the node.
\item Nonterminal nodes are written
as \texttt{ITE(V$x$, N$h$, N$l$)}, where $x$ indicates the variable,
and \texttt{N}$h$ and \texttt{N}$l$ indicate the two children.
\end{itemize}

\begin{choice}
\item
The following nodes each correspond to conjunctions of some set of the
input clauses.  Fill in the following table with that information, as
is illustrated with the first entry.
\begin{center}
\renewcommand{\arraystretch}{1.1}
\begin{tabular}{cl}
\toprule
\makebox[.5in]{Node} & \makebox[1.5in]{Clauses} \\
\midrule
\texttt{N16} & \texttt{1}, \texttt{2} \\
\texttt{N17} & \squick{\texttt{3}, \texttt{4}} \\
\texttt{N18} & \squick{\texttt{5}, \texttt{6}} \\
\texttt{N21} & \squick{\texttt{1}, \texttt{2}, \texttt{3}, \texttt{4}} \\
\bottomrule
\end{tabular}
\end{center}

\item
Draw the BDD having node \texttt{N21} as root.
\solution{1.5in}{\begin{center}\input{dd/n21.tex}\end{center}}
\item
Looking at the paths from the root to Leaf 1 in this BDD, what
constraint do these place on the relation between  $a$ and $b$?
\solution{1in}{The only two paths require that $a=b$.}
\item
What does that constraint imply when the conjunction of the BDDs with roots \texttt{N21} and \texttt{N18} is formed?
\solution{1in}{
The BDD with root \texttt{N18} is a representation of $a \oplus b$,
and so it requires $a \not = b$.  Their conjunction is
therefore $\bot$.
}
\end{choice}
\end{problem}

\newpage

\begin{problem}{I}
When running TBSAT on \texttt{cyc3.cnf} in bucket mode (see Slide~\#16--17 of Lecture 2b), the generated proof contains these comments.
The full proof is in file \texttt{files/cyc3-bkt.lrat}.

\begin{lstlisting}
c Input Clause #1: 2 1 0
c Input Clause #2: -2 -1 0
c Input Clause #3: 3 2 0
c Input Clause #4: -3 -2 0
c Input Clause #5: 3 1 0
c Input Clause #6: -3 -1 0
c Defining clauses for node N4 = ITE(V1 (level=1), N1, N0)
c Defining clauses for node N5 = ITE(V1 (level=1), N0, N1)
c Defining clauses for node N6 = ITE(V2 (level=2), N1, N0)
c Defining clauses for node N7 = ITE(V2 (level=2), N0, N1)
c Defining clauses for node N8 = ITE(V3 (level=3), N1, N0)
c Defining clauses for node N9 = ITE(V3 (level=3), N0, N1)
c Defining clauses for node N10 = ITE(V1 (level=1), N1, N6)
c Validate BDD representation of Clause #1.  Node = N10.
c Defining clauses for node N11 = ITE(V1 (level=1), N7, N1)
c Validate BDD representation of Clause #2.  Node = N11.
c Defining clauses for node N12 = ITE(V2 (level=2), N1, N8)
c Validate BDD representation of Clause #3.  Node = N12.
c Defining clauses for node N13 = ITE(V2 (level=2), N9, N1)
c Validate BDD representation of Clause #4.  Node = N13.
c Defining clauses for node N14 = ITE(V1 (level=1), N1, N8)
c Validate BDD representation of Clause #5.  Node = N14.
c Defining clauses for node N15 = ITE(V1 (level=1), N9, N1)
c Validate BDD representation of Clause #6.  Node = N15.
c Defining clauses for node N16 = ITE(V1 (level=1), N7, N6)
c Generating proof that N10 & N11 --> N16
c Defining clauses for node N17 = ITE(V1 (level=1), N9, N8)
c Generating proof that N14 & N15 --> N17
c Defining clauses for node N18 = ITE(V2 (level=2), N8, N0)
c Defining clauses for node N19 = ITE(V2 (level=2), N0, N9)
c Defining clauses for node N20 = ITE(V1 (level=1), N19, N18)
c Generating proof that N16 & N17 --> N20
c Defining clauses for node N21 = ITE(V2 (level=2), N8, N9)
c Generating proof that N20 --> N21
c Defining clauses for node N22 = ITE(V2 (level=2), N9, N8)
c Generating proof that N12 & N13 --> N22
c Validate empty clause for node N0 = N21 & N22
\end{lstlisting}

\newpage
\begin{choice}
\item Which clauses are conjuncted to form the BDD with root node \texttt{N20}?  What does this BDD represent?
\solution{1in}{Clauses \texttt{1}, \texttt{2}, \texttt{5}, and \texttt{6}.
These are the clauses containing the top variable $a$. This BDD is a conjunction of the BDDs for Bucket A\@.}

\item Draw a diagram of the BDD with root node \texttt{N20}.
\solution{1.5in}{\begin{center}\input{dd/n20.tex}\end{center}}

\item The BDD with root node \texttt{N21} is the result of existentially quantifying $a$ from \texttt{N20}, and
the BDD with root node \texttt{N22} is the result of conjuncting clauses $C_3$ and $C_4$.  
Draw a diagram showing both of these BDDs together.
\solution{1.5in}{\begin{center}\input{dd/n21-22.tex}\end{center}}

\item What happens when the conjunction of the BDDs with root nodes \texttt{N21} and \texttt{N22} is formed?  Explain
\solution{1.0in}{\texttt{N21} requires $b$ and $c$ to have even parity, while \texttt{N22} requires them to have odd parity.
Their conjunction is therefore $\bot$.}

\end{choice}
\end{problem}

\newpage
\section*{Proof Generation with BDDs}

The following shows a portion of the LRAT proof file generated when TBSAT is applied to \texttt{cyc3.cnf}.
The full proof is in file \texttt{files/cyc3-dct.lrat}.
\begin{lstlisting}
c Input Clause #1: 2 1 0

c Defining clauses for node N6 = ITE(V2 (level=2), N1, N0)
15 6 -2 0  0
18 -6 2 0 -15 0

c Defining clauses for node N10 = ITE(V1 (level=1), N1, N6)
31 10 -1 0  0
32 10 -6 1 0  0
34 -10 6 1 0 -31 -32 0
c Validate BDD representation of Clause #1.  Node = N10.
35 10 0 31 32 15 1 0
\end{lstlisting}

\begin{problem}{I}
As the comments indicate, clauses \texttt{15}, \texttt{18}, and \texttt{31}--\texttt{34} are defining
clauses for two of the nodes.  With LRAT, extension variables are
introduced via such clauses, where the literal for the extension variable is listed
first.  Rather than the antecedents that we saw in the RUP clause
examples of the lectures, the second portion of a defining clause
lists any previous clauses containing the opposite literal of that of
the extension variable, but these are also negated.  So for example,
Clause \texttt{18} lists \texttt{-15}, while clause \texttt{34}
lists \texttt{-31} and \texttt{-32}.  (Understanding this part of the
syntax is not critical here.)

\begin{choice}
\item In general, there can be up to four defining clauses per BDD node, as was shown on Slide~\#4 of Lecture~2b.
But, when a clause degenerates to a tautology, it is not included in
the proof.
Show how clauses \texttt{15} and \texttt{16} match up to the defining clauses for node \texttt{N6}.
\solution{0.5in}{
Clause \texttt{15} represents \proofhu{N6}, and
clause \texttt{18} represents \proofld{N6}.
Clauses \prooflu{N6} and \proofhd{N6} are tautologies.
}

\item
Show how clauses \texttt{31}, \texttt{32}, and \texttt{34} match up to the defining clauses for node \texttt{N10}.
\solution{0.5in}{
Clause \texttt{31} represents \proofhu{N10},
clause \texttt{32} represents \prooflu{N10},
and clause \texttt{34} represents \proofld{N10}.
Clause \proofhd{N10} is a tautology.
}

\item
Simulate the RUP proof steps to provide a justification of
unit clause \texttt{35}, indicating that node \texttt{N10} is the TBDD representation of input clause \texttt{1}.
\solution{0.5in}{
Starting with literal \texttt{-10},
clause \texttt{31} generates unit \texttt{-1}, and \texttt{32}
generates unit \texttt{-6}.  Clause \texttt{15} then generates
unit \texttt{-2}, while clause \texttt{1} generates a conflict.
}
\end{choice}
\end{problem}

\newpage
The following shows a portion of the LRAT proof file generated when TBSAT is applied to \texttt{cyc3.cnf} in direct mode.
The full proof is in file \texttt{files/cyc3-dct.lrat}.
\begin{lstlisting}
c Input Clause #1: 2 1 0
c Input Clause #2: -2 -1 0

c Defining clauses for node N6 = ITE(V2 (level=2), N1, N0)
15 6 -2 0  0
18 -6 2 0 -15 0
c Defining clauses for node N7 = ITE(V2 (level=2), N0, N1)
20 7 2 0  0
21 -7 -2 0 -20 0

c Defining clauses for node N10 = ITE(V1 (level=1), N1, N6)
31 10 -1 0  0
32 10 -6 1 0  0
34 -10 6 1 0 -31 -32 0
c Validate BDD representation of Clause #1.  Node = N10.
35 10 0 31 32 15 1 0

c Defining clauses for node N11 = ITE(V1 (level=1), N7, N1)
36 11 -7 -1 0  0
37 11 1 0  0
38 -11 7 -1 0 -36 -37 0
c Validate BDD representation of Clause #2.  Node = N11.
40 11 0 37 36 20 2 0

c Defining clauses for node N16 = ITE(V1 (level=1), N7, N6)
61 16 -7 -1 0  0
62 16 -6 1 0  0
63 -16 7 -1 0 -61 -62 0
64 -16 6 1 0 -61 -62 0
c Generating proof that N10 & N11 --> N16
65 16 -11 -10 -1 0 61 38 0
66 16 -11 -10 0 65 62 34 0

c Validate unit clause for node N16 = N10 & N11
67 16 0 35 40 66 0
\end{lstlisting}

\newpage

\begin{choice}
\item
Proof clauses \texttt{65} and \texttt{66} justify that
node \texttt{N16} is the conjunction of BDDs representing input
clauses \texttt{1} and \texttt{2}.  It is a special case of the
general form shown on Slide~\#7 of Lecture~2b, where the
justifications for the two recursive calls are the tautologies
$b \land 1 \rightarrow b$ and $\overline{b} \land 1 \rightarrow \overline{b}$.
Match up the other clauses to those shown on Slide \#7.

\solution{1in}{
Clause \texttt{61} represents \proofhu{N16}.
Clause \texttt{38} represents \proofhd{N11}.
Clause \texttt{62} represents \prooflu{N16}.
Clause \texttt{34} represents \proofld{N20}.
}

\item 
Simulate the RUP proof steps to justify proof clauses \texttt{65} and \texttt{66}.

\solution{1in}{
The proof for clause \texttt{65} starts with literals \texttt{-16}, \texttt{11}, \texttt{10}, and \texttt{1}. Clause \texttt{61} gives \texttt{-7}.
Clause Clause \texttt{38} gives conflict.

The proof for clause \texttt{66} starts with literals \texttt{-16},
\texttt{11}, and
\texttt{10}.
Clause
\texttt{65}
gives
\texttt{-1}.
Clause
\texttt{62}
gives
\texttt{-6}.
Clause
\texttt{34}
gives conflict.
}

\item
What is the significance of proof clause \texttt{67}?  What is its justification?
\solution{1in}{
This step justifies the unit clause for \texttt{N16}, showing that \texttt{N16} is the TBDD representation for the conjunction of input clauses \texttt{1} and \texttt{2}.  It follows from 1) the unit clauses for the BDDs representing the two clauses and 2) the proof for the ApplyAnd operation on step \texttt{66}.
}

\end{choice}

\newpage
\section*{Minimal Disagreement Parity}

\newcommand{\ints}[1]{{\cal N}_{#1}}
\newcommand{\bset}{{\cal B}}


For the remainder of the lab, you will test SAT solvers on a problem
involving a combination of parity reasoning and cardinality
constraints.

Crawford, Kearns, and Shapire proposed the Minimum Disagreement Parity (MDP)
Problem as a challenging SAT benchmark in an unpublished report from
AT\&T Bell Laboratories in 1994.  The report is available at:
\begin{center}
\hcref{http://www.cs.cornell.edu/selman/docs/crawford-parity.pdf}.
\end{center}

MDP is closely related to the ``Learning Parity with Noise'' (LPN)
problem.  LPN has been proposed as the basis for public key
crytographic systems.  Unlike the widely used
RSA cryptosystem, it is resistant to all known quantum
algorithms.  The capabilities of SAT
solvers on MDP is therefore of interest to the cryptology community.

The program \texttt{generators/mdparity.py} generates CNF formulas
encoding instances of the MDP problem.

In the following, let $\bset = \{0, 1\}$ and $\ints{p} = \{1, 2, \ldots, p\}$.
Assume all arithmetic is performed modulo 2.  Thus,
if $a, b \in \bset$, then $a+b \equiv a \oplus b$.

The problem is parameterized by a number of solution bits $n$, a
number of samples $m$, and an error tolerance $k$, as follows.  Let
$\bm{s} = s_1, s_2, \ldots, s_n$ be a set of {\em solution} bits.
For $1 \leq j \leq m$, let $X_j \subseteq \ints{n}$
be a {\em sample set}, created by generating $n$
random bits $x_{1,j}, x_{2,j}, \ldots x_{n,j}$
and letting $X_j = \{i | x_{i,j} = 1 \}$.  Let $\bm{y} = y_1, y_2, \ldots, y_m$ be the
parities of the solution bits for each of the $m$ samples:
\begin{eqnarray}
y_j & = & \sum_{i \in X_j} s_i \label{eqn:uncorrupted}
\end{eqnarray}

Given sufficiently many samples $m$ for there to be at least $n$ linearly
independent sample sets, the values of the solution bits $\bm{s}$ can
be uniquely determined from $\bm{y}$ and the sample sets $S_j$ for
$1 \leq j \leq m$ by Gaussian elimination.  To make this problem
challenging, we introduce ``noise,'' allowing up to $k$ of these
samples to be ``corrupted'' by flipping the values of their parity.
That is, let $T \subseteq \ints{m}$ be created by randomly choosing
$k$ values from $\ints{m}$ without replacement, and define $m$
``corruption'' bits $\bm{r} = r_1, r_2, \ldots, r_m$, with $r_j$ equal
to $1$ if $j \in T$ and equal to $0$ otherwise.  
We then provide noisy samples $\bm{y}'$, defined as:
\begin{eqnarray}
y'_j & = & r_j + \sum_{i \in X_j} s_i \label{eqn:corrupted}
\end{eqnarray}
and require the correct solution bits $\bm{s}$ to be determined despite this noise.  That is, the generated solution $\bm{s}$ must satisfy at least $m-k$ of equations (\ref{eqn:uncorrupted}).
For larger values of $k$, the
problem becomes NP-hard.

We can see the potential of this problem for cryptographic
applications.  A set of $m$ parity constraints over $n$ variables serves as a ``lock'',
where the ``key'' is a set of $n$ solution bits. 
Checking that
the key fits the lock requires simply checking that it satisfies at least $m-k$ of the parity constraints.
The lock can be encoded
with $O(m \cdot n)$ bits, and the key is just $m$ bits.

This problem can readily be encoded in CNF with variables for
unknown values $\bm{s}$ and $\bm{r}$, along with some auxilliary
variables.  Each of the $m$ equations
(\ref{eqn:corrupted}) is encoded using auxilliary variables to avoid
exponential expansion.  An at-most-$k$ constraint is imposed on the
corruption bits $\bm{r}$.

Crawford and his colleagues suggests choosing $n$ to be a multiple of
4 and letting $m = 2n$ and $k = m/8 = n/4$, but the tools they had to
investigate these parameters didn't include today's powerful SAT
solvers.

From the perspective of SAT solving, MDP has an interesting form.
Structurally, it consists of a set of $m$ parity constraints over a
set of $n+m$ variables---the $n$ solution bits $\bm{s}$, plus the $m$
corruption bits $\bm{r}$, with each constraint $j$ having the form
shown in (\ref{eqn:corrupted}).  Additionally, there is an at-most-$k$
constraint imposed on the corruption bits.  Performing Gaussian
elimination on the parity constraints reduces them to a smaller set of
constraints over just the corruption bits.  The problem to be solved
then becomes: ``Given a set of parity constraints over $m$ variables,
is there a solution for which at most $k$ of these variables are set
to 1?''  That is not a problem that can be solved by algebraic
methods.

Thus, Gaussian elimination can greatly simplify the constraints, but
unlike other parity constraint benchmarks (e.g., the Chew-Heule
formulas), it must be combined with other SAT solving techniques.
KISSAT does not do anything special with parity constraints---it will
apply CDCL to the original problem.  TBSAT, on the other hand, can be
directed to first apply Gaussian elimination and then perform bucket
elimination.

\subsection*{Experiments to Perform}

An instance of the problem is guaranteed to have one solution, but it
might also have multiple solutions, possibly weakening its
cryptographic properties.  The chances of the solution being unique
can be improved by increasing $m$ relative to $n$, but it would be
good to avoid taking this to an extreme.

To enable checking for uniqueness, the benchmark generator has an
option `\texttt{-x}' that causes the generator to insert a clause that
excludes the nominal solution.  The resulting formula will then be
unsatisfiable if there are no other solutions.

The directory \texttt{mdp-run} is set up for generating instances of
the benchmark (with the nominal solution excluded) and running both
KISSAT and TBSAT, with and without proof generation via different
options in the Makefile.  The key parameters to be set are $n$, $m$,
$k$, and the random seed.  When running without proof generation, the
Makefile also invokes a solution checker.  This program reads the
comments in the CNF file to determine the parameters of the problem
instance and then checks the generated SAT solution to ensure that it
is a valid solution to the problem.  The Makefile is also configured
to have the solvers time out if they run for more than 600 seconds.

\begin{problem}{II}
\plabel{prob08}
The bash script \texttt{g08.sh} executes runs without proof generation
for $n=8$.

Try running this script.  It will produce a number of data files.  You
can then use \texttt{grep} to check results for the runs.  Here are some useful patterns to search:
\begin{center}
\begin{tabular}{lll}
\toprule
Program & Search Pattern & Result \\
\midrule
Either & \texttt{"SATISFIABLE"} & Whether or not the formula is satisfiable \\
Either & \texttt{"SOLUTION VERIFIED"} & Whether the generated solution is valid \\
\midrule
TBSAT & \texttt{"Performing Gauss-Jordan"} & Information from Gaussian elimination phase \\
TBSAT & \texttt{"Gauss-Jordan completed"} & Information from Gaussian elimination phase \\
TBSAT & \texttt{"Total clauses"} & Size of the proof (if generated) \\
TBSAT & \texttt{"Elapsed seconds"} & TBSAT runtime \\
TBSAT & \texttt{"Timeout"} & Program timed out \\
\midrule
KISSAT & \texttt{"proof\_added"} & Size of the proof (if generated) \\
KISSAT & \texttt{"process-time"} & KISSAT runtime \\
\bottomrule
\end{tabular}
\end{center}

Using those search terms, determine the following:
\begin{choice}
\item How did the number of satisfiable instances change as $m$ increased?
\item Were all of the solutions for the satisfiable instances valid?
\item With TBSAT, how effective was Gaussian elimination in reducing the number of parity consraints?
\item How did the runtimes of KISSAT and TBSAT compare?
\end{choice}
\end{problem}

\newpage

\begin{problem}{II}
The bash script \texttt{g24.sh} executes runs without proof generation
for $n=24$.

Run this script an perform the same evaluations as you did for $n=8$ in Problem~\pref{prob08}.
\end{problem}

\begin{problem}{III}

Consider other experiments on the MDP problem that you can perform using the availabile tools.  Some possible parameters to explore are as follows
\begin{center}
\begin{tabular}{rrr}
\toprule
\makebox[0.75in]{$n$} &
\makebox[0.75in]{$m$} &
\makebox[0.75in]{$k$} \\
\midrule
8 & 16 & 2 \\
8 & 24 & 3 \\
\midrule
16 & 32 & 4 \\
16 & 40 & 5 \\
16 & 48 & 6 \\
\midrule
24 & 48 & 6 \\
24 & 56 & 7 \\
24 & 64 & 8 \\
\midrule
28 & 56 & 7 \\
28 & 64 & 8 \\
28 & 72 & 9 \\
\midrule
32 & 64 & 8 \\
32 & 72 & 9 \\
32 & 80 & 10 \\
\bottomrule
\end{tabular}
\end{center}

Some things you could evaluate include:
\begin{choice}
\item How does the number of satisfiable instances vary with $n$ and $m$?
\item How do the two SAT solvers perform on these problems?
\item For the unsatisfiable instances, how large are the proofs when proof generation is enabled?
\item How does the runtime change for the two solvers when proof generation is enabled?
\end{choice}

\end{problem}


\end{document}


