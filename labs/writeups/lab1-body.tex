
\usepackage[pdftex]{graphicx}


\usepackage{exercise}
\usepackage{enumitem}
\usepackage{color}
\usepackage{booktabs}

\newcommand{\pathname}[2]{\textit{ROOT}\texttt{/}\texttt{#1}\texttt{/}\texttt{#2}}

\begin{document}
\begin{center}
{\LARGE\bf Summer School on Formal Techniques\\ [2 ex]
\showexamname{1}}\\ [2 ex]
June, 2022
\end{center}
\section*{Explanation}

These exercises are designed to provide a deeper understanding of the
operation of Boolean satisfiability (SAT) solvers, especially when
applied to unsatisfiable formulas.  A key requirement is that solver
be able to generate a proof of unsatisfiability in such cases.

The provided problems range in how much time and effort is required,
and whether any programming is involved.  Each problem has an
associated {\em level}, according to the following standard:
\begin{description}
\item[I:] Simple pencil-and-paper exercises designed to provide a
  concrete examples for the concepts presented.  Doing these will
  help you gain confidence in the concepts being presented
\item[II:] More challenging pencil-and-paper exercises, or algorithmic
  and experimental activities.  These may running solvers on some
  benchmarks.
\item[III:] Deeper explorations.  These may require devising new
  algorithms, writing code, and performing experiments that go beyond
  the core lecture material.
\end{description}

All file names are specified in this document are given as path names
of the form
\pathname{\textit{DIR}}{\textit{FILE}}
where
\textit{ROOT} indicates the root of the directory structure,
\textit{DIR} is either ``\texttt{files}'' or ``\texttt{generators},''
and \textit{FILE} is the file name.


\newpage

\section*{Using the Provided Programs}

Here is how to use the provided tools

\subsection*{KISSAT}

\begin{itemize}
\item Running without proof generation
\item[]\pathname{kissat/build}{kissat} \texttt{--no-binary} {\it FORMULA}\texttt{.cnf}
\item Running with proof generation
\item[]\pathname{kissat/build}{kissat} \texttt{--no-binary} {\it FORMULA}\texttt{.cnf} {\it PROOF}\texttt{.drat}
\end{itemize}

\subsection*{TBSAT}

\begin{itemize}
\item Running in direct mode without proof generation
\item[]\pathname{tbuddy/src/tbsat}{tbsat} \texttt{-i} {\it FORMULA}\texttt{.cnf}
\item Running in direct mode with proof generation
\item[]\pathname{tbuddy/src/tbsat}{tbsat} \texttt{-i} {\it FORMULA}\texttt{.cnf} \texttt{-o} {\it PROOF}\texttt{.lrat}
\item Running in bucket mode with proof generation
\item[]\pathname{tbuddy/src/tbsat}{tbsat} \texttt{-b} \texttt{-i} {\it FORMULA}\texttt{.cnf} \texttt{-o} {\it PROOF}\texttt{.lrat}
\end{itemize}


\subsection*{DRAT-TRIM}

\begin{itemize}
\item Checking proof
\item[]\pathname{drat-trim/build}{drat-trim} {\it FORMULA}\texttt{.cnf} {\it PROOF}\texttt{.drat}
\item Transforming a DRAT proof into an LRAT proof
\item[]\pathname{drat-trim/build}{drat-trim} {\it FORMULA}\texttt{.cnf} {\it PROOF}\texttt{.drat} \texttt{-L} {\it PROOF}\texttt{.lrat}
\end{itemize}

\subsection*{LRAT-CHECK}

\begin{itemize}
\item Checking proof
\item[]\pathname{drat-trim/build}{lrat-check} {\it FORMULA}\texttt{.cnf} {\it PROOF}\texttt{.lrat}
\end{itemize}



\newpage

\section*{The Pigeonhole Problem PHP($n$)}


The formula PHP($n$) encodes the impossible problem of assigning $n+1$
pigeons to $n$ holes such that 1) each pigeon is assigned to some hole
and 2) no hole contains more than one pigeon.  It is described in
Slide~\#14 of Lecture 1b.  The SAT encoding uses variables $p_{i,j}$
for $1 \leq i \leq n$ and $1 \leq j \leq n+1$, where $p_{i,j}$ is 1
when pigeon $j$ is assigned to hole $i$.  It makes use of
the cardinality constraints described in Slide~\#18 of Lecture 1a.
\medskip

\begin{problem}{I}
File \pathname{files}{phpd-02.cnf} contains
a DIMACS representation of the CNF encoding of PHP(2).
It consists of the following lines
\begin{lstlisting}
p cnf 6 9
1 4 0
2 5 0
3 6 0
-1 -2 0
-1 -3 0
-2 -3 0
-4 -5 0
-4 -6 0
-5 -6 0
\end{lstlisting}

\begin{choice}
\item Describe how each variable $p_{i,j}$ is mapped to a variable number in the file.
  \solution{1in}{They are in row-major
    order.  $p_{i,j}$ is numbered $(n+1)\cdot (i-1) + j = 3 \cdot (i-1) + j$.}

\item Describe how the following sets of clauses encode the problem:

\ifshowsolutions
$\;$\\[1ex]{\bf ANSWER:}
  \begin{description}
    \item[Clauses 1--3]: These encode the at-least-one constraints for pigeons 1--3
    \item[Clauses 4--6]: This encodes the at-most-one constraint for hole 1
    \item[Clauses 7--9]: This encodes the at-most-one constraint for hole 2
  \end{description}
\else
  \begin{description}
    \item[Clauses 1--3]:
    \item[Clauses 4--6]:
    \item[Clauses 7--9]:
  \end{description}
\fi
\end{choice}
\end{problem}

\newpage

\begin{problem}{II}
In this exercise, you will determine the number of clauses in the
encoding of PHP($n$) according to a {\em direct} encoding of the
at-most-one constraints, as is described on Slide~\#18 of Lecture 1a.

\begin{choice}
\item How many at-least-one constraints are required?  How many clauses does each require?
  \solution{1in}{$n+1$ constraints, each  $1$ clause}

\item How many at-most-one constraints are required?  How many clauses does each require?
  \solution{1in}{$n$ constraints, each $(n+1)\cdot n/2$ clauses}
  
\item What is the total number of clauses required to encode PHP($n$)?  You can write this as an expression of the form $\approx a \cdot n^b$, meaning that the exact count is an expression of the form
  $a \cdot n^b + o(n^b)$.

  \solution{1in}{$n + 1 + (n+1)\cdot n^2/2 \approx 1/2 \cdot n^3$}

\end{choice}
\end{problem}

\newpage

\begin{problem}{III}
\plabel{sinz}

The direct encoding of at-most-one (AMO) constraints shown on
Slide~\#18 of Lecture 1a scales quadratically with the number of
variables.  While polynomial, this can become unwieldy for large
values of $n$.  Here we will consider a method to reduce the size of
the encoding to $O(n)$ by using auxilliary variables, analogous to their
use in encoding parity constraints, as is described on Slides~\#16--17
of Lecture~1a.  Our goal is to derive a method described in a paper by Carsten Sinz in 2005.

For $n > 2$, let us encode ${\it AMO}(x_1, x_2, \ldots, x_n)$ by the following process:
  \begin{enumerate}
  \item Introduce a new variable $z$
  \item Encode some set of constraints ${\it LCON}(x_1, x_2, z)$
  \item Recursively encode ${\it AMO}(z, x_3, \ldots, x_n)$.
  \item The recursion terminates with a constraint of the form
    ${\it  RCON}(z', x_n)$, where $z'$ was the final variable added.
  \end{enumerate}

\begin{choice}
\item
    What constraints should be encoded as ${\it LCON}$?  How would these be expressed as clauses?
    \solution{0.5in}{$z$ should be set to 1 when either $x_1$ or $x_2$ is
      1.
      In addition, $x_2$ should be set to 0 when $x_1$ is one.  This leads to the following three clauses:
      $(\overline{x}_1 \lor z)$, $(\overline{x}_2 \lor z)$, $(\overline{x}_1 \lor \overline{x}_2)$.
      }

\item
   What constraints should be encoded as ${\it RCON}$?  How would these be expressed as clauses?   
   \solution{0.5in}{$x_n$ should be set to 0 when $z'$ is 1.
     This is expressed with the clause $(\overline{z}' \lor \overline{x}_n)$.}

\item
  Show the set of clauses that this process would generate for ${\it AMO}(x_1, x_2, x_3, x_4, x_5)$
\solution{0.75in}{
    This would require three new variables $z_2$, $z_3$, and $z_4$ and ten clauses:
\begin{center}
    \begin{tabular}{|c|c|c|c|}
      \hline
      ${\it LCON}(x_1, x_2, z_2)$ &
      $\overline{x}_1 \lor z_2$ &  $\overline{x}_2 \lor z_2$ & $\overline{x}_1 \lor \overline{x}_2$ \\
      \hline
      ${\it LCON}(z_2, x_3, z_3)$ &
      $\overline{z}_2\lor z_3$ &  $\overline{x}_3 \lor z_3$ & $\overline{z}_2 \lor \overline{x}_3$ \\
      \hline
      ${\it LCON}(z_2, x_4, z_4)$ &
      $\overline{z}_3\lor z_4$ &  $\overline{x}_4 \lor z_4$ & $\overline{z}_3 \lor \overline{x}_4$ \\
      \hline
      ${\it RCON}(z_4, x_5)$ &
      && $\overline{z}_4 \lor \overline{x}_5$ \\
      \hline
    \end{tabular}
\end{center}
}

\item
  For $n \geq 3$, how many additional variables and how many clauses does this encoding require?
  \solution{0.75in}{$n-1$ variables and $3(n-2)+1$ clauses}
    
\item
  How many total variables and how many total clauses would this
  method require to encode PHP($n$)?  As before, you can write these as
  expressions of the form $\approx a\cdot n^b$.
  \solution{0.5in}{
    There are still $\approx n^2$ problem variables,
    and now there are $\approx n^2$ additional variables, for a total of $\approx 2\cdot n^2$.
    There are $\approx n$ clauses for the ALO constraints and $\approx
    n$ AMO constraints, each with $\approx 3\cdot n$ clauses, yielding a total of $\approx 3\cdot n^2$ clauses.
  }

\end{choice}
\end{problem}

\newpage

\section*{CDCL Operation}

In the following problems, you will simulate the behavior of a CDCL
solver by hand.  Pseudocode for the algorithm is given on Slide~\#6 of
Lecture~1b.  Rather than diagramming the execution as was done on
Slide~\#7, you can simply write a sequence of literals describing one
execution of inner loop of the algorithm.  Use some method (e.g.,
colors, underlining) to distinguish literals that are set by unit
propagation versus those assigned by choice.  A conflict occurs when
the same variable has been assigned both 1 and 0.  Then finish the
line with the generated conflict clause.

\definecolor{xred}{rgb}{0.77, 0.12, 0.23}
\newcommand{\rtext}[1]{\textcolor{xred}{#1}}


For example, we would use the following notation to describe the
execution shown on Slide\#7, where assigned literals are indicated in
red.
\begin{center}
  \begin{tabular}{l|l}
    \toprule
    Sequence & Conflict Clause \\
    \midrule
    $\rtext{\overline{b}} \; \overline{d} \; d$ & $b$ \\
    $b \; \rtext{c} \; \overline{a} \; \overline{d} \; d$ & $\overline{c}$ \\
    $b \; \overline{c} \; \overline{a} \; \overline{d} \; d$ & $\bot$ \\
    \bottomrule
   \end{tabular}
\end{center}

To make the process deterministic, follow these conventions:
\begin{enumerate}
\item When choosing a variable and an assignment, choose the least
  numbered unassigned variable and assign it value 1.
\item Perform unit propagations in breadth-first order, and for each
  of these in the order of the clauses.  That is, when processing some
  some literal in your sequence, do a pass over the clauses, adding
  any unit propagations to the end of your sequence.
\end{enumerate}

\newpage

\begin{problem}{I}
\plabel{php2sim}
  Show how CDCL would execute when following these conventions on the following
  DIMACS file, encoding PHP(2).

  This file is available as \pathname{file}{phpd-02.cnf}
\begin{lstlisting}
p cnf 6 9
1 4 0
2 5 0
3 6 0
-1 -2 0
-1 -3 0
-2 -3 0
-4 -5 0
-4 -6 0
-5 -6 0
\end{lstlisting}
  You can use the DIMACS conventions for writing literals and clauses.
  
\solution{1.5in}{
\begin{center}
  \begin{tabular}{l|l}
    \toprule
    Sequence & Conflict Clause \\
    \midrule
    \texttt{\rtext{1} -2 -3 5 6 -4 -6} & \texttt{-1 0} \\
    \texttt{-1 4 -5 -6 2 3 -3} & \texttt{0} \\
    \bottomrule
    \end{tabular}
\end{center}
}
\end{problem}


\begin{problem}{I}
\plabel{eg1sim}
  Show how CDCL would execute when following these conventions on the following
  DIMACS file, encoding the example formula from Lectures 2a and 2b.

  This file is available as \pathname{file}{eg-1.cnf}  
  \begin{lstlisting}
p cnf 4 6
-1 -2 -3    0
-1 -2  3    0
 1       -4 0
 1        4 0
    2    -4 0
cd     2     4 0
  \end{lstlisting}
  You can use the DIMACS conventions for writing literals and clauses.
  
\solution{1.5in}{
\begin{center}
  \begin{tabular}{l|l}
    \toprule
    Sequence & Conflict Clause \\
    \midrule
    \texttt{\rtext{1} \rtext{2} -3 3} & \texttt{-1 -2 0} \\
    \texttt{\rtext{1} -2 -4 4} & \texttt{-1 0} \\
    \texttt{-1 -4 4} & \texttt{0} \\
    \bottomrule
    \end{tabular}
\end{center}
}
\end{problem}

\begin{problem}{II}
\plabel{php3sim}
  Show how CDCL would execute when following these conventions on the following
  DIMACS file, encoding PHP(3).

  This file is available as \pathname{file}{phpd-03.cnf}
\begin{lstlisting}
p cnf 12 22
1 5 9 0
2 6 10 0
3 7 11 0
4 8 12 0
-1 -2 0
-1 -3 0
-1 -4 0
-2 -3 0
-2 -4 0
-3 -4 0
-5 -6 0
-5 -7 0
-5 -8 0
-6 -7 0
-6 -8 0
-7 -8 0
-9 -10 0
-9 -11 0
-9 -12 0
-10 -11 0
-10 -12 0
-11 -12 0
\end{lstlisting}
  You can use the DIMACS conventions for writing literals and clauses.
  
\solution{1.5in}{
\begin{center}
  \begin{tabular}{l|l}
    \toprule
    Sequence & Conflict Clause \\
    \midrule
  \texttt{\rtext{1} -2 -3 -4 \rtext{5} -6 -7 -8 10 11 12 -11} & \texttt{-1 -5 0} \\  
  \texttt{\rtext{1} -5 9 \rtext{6} -7 -8 11 -9} & \texttt{-1 -6 0} \\  
  \texttt{\rtext{1} -5 -6 9 -10 10} & \texttt{-1 0} \\  
  \texttt{-1 \rtext{2} -3 -4 \rtext{5} -6 -7 -8 11 12 -11} & \texttt{-2 -5 0} \\  
  \texttt{-1 \rtext{2} -3 -4 -5 9 -10 -11 -12 7 8 -8} & \texttt{-2 0} \\  
  \texttt{-1 -2 \rtext{3} -4 \rtext{5} -6 -7 -8 10 12 -12} & \texttt{-3 -5 0} \\  
  \texttt{-1 -2 \rtext{3} -4 -5 9 -10 -11 -12 8 -7 -6 11} & \texttt{-3 0} \\  
  \texttt{-1 -2 -3 4 \rtext{5} -6 -7 -8 10 -9 -12 11} & \texttt{-5 0} \\
  \texttt{-1 -2 -3 4 -5 9 -10 -11 -12 6 7 8 -7} & \texttt{0} \\
  \bottomrule
    \end{tabular}
\end{center}
}
\end{problem}

\newpage

\section*{CDCL Proof Generation}


\begin{problem}{I}
Create a DRAT file containing the conflict clauses you generated in your solution to Problem~\pref{php2sim}.
Check that it is a valid proof using DRAT-TRIM.
\end{problem}

\begin{problem}{I}
Create a DRAT file containing the conflict clauses you generated in your solution to Problem~\pref{eg1sim}.
Check that it is a valid proof using DRAT-TRIM.
\end{problem}

\begin{problem}{II}
Create a DRAT file containing the conflict clauses you generated in your solution to Problem~\pref{php3sim}.
Check that it is a valid proof using DRAT-TRIM.
\end{problem}

\newpage

\section*{Experimenting with PHP($n$)}

In the following problems, you will explore how proofs of PHP($n$)
scale with $n$ when running existing SAT solvers.  You will evalute
different solvers and different encodings of the formula.

The file \pathname{generators}{gen\_pigeon.py} can be used to generate
instances of pigeonhole formulas.  Here are its command-line options:
\begin{description}
\item[\texttt{-h}:] Print documentation
\item[\texttt{-v}:] Verbose mode.
  The generator will put comments in
  the CNF file describing how the problem is encoded.  You may find
  these instructive.
\item[\texttt{-L}:] Generate a linear encoding of the at-most-one
  constraints according to the solution you devised for
  Problem~\pref{sinz}.  {\bf WARNING:} You must first add code
  to the generator before this will work.
\item[\texttt{-r \textit{ROOT}}:]
  Specify root name of output file.  For example, if you specify the
  root ``\texttt{PHP},'' the generated file will be named
  \texttt{PHP.cnf}.
\item[\texttt{-n $N$}:]
  Specify the number of holes
\item[\texttt{-p $P$}:]
  Specify the number of pigeons.  The default is to have $P=N+1$,
\end{description}

\begin{problem}{II}
  Generate direct encodings of PHP($n$) and run KISSAT to fill in the
  following table.  You can get the number of input variables and clauses from the header of the CNF file.
To get the number of proof clause, you must run KISSAT giving the name of the proof file
  on the command line.  The output line labeled
  ``\texttt{proof\_added}'' shows the number of proof clauses.
  What can you infer about how the proof size scales?
\begin{center}
\ifshowsolutions
    \renewcommand{\arraystretch}{1.1}
\else
    \renewcommand{\arraystretch}{1.2}
\fi
    \begin{tabular}{rrrr}
\toprule
\makebox[0.75in]{$n$} & \makebox[0.75in]{Input} & \makebox[0.75in]{Input}
& \makebox[1.5in]{Proof} \\
\makebox[0.75in]{} & \makebox[0.75in]{Variables} & \makebox[0.75in]{Clauses}
& \makebox[1.5in]{Clauses} \\
\midrule
\ifshowsolutions
      4 & 20  & 45 & 49 \\
      6 & 42 & 133 & 1,053 \\
      8 & 72 & 297 & 45,691 \\
      10 & 110 & 561 & 4,242,435 \\
      11 & 132 & 738 & 40,030,414 \\
\else
      4 & & & \\
\midrule
      6 & &  &  \\
\midrule
      8 & & &  \\
\midrule
      10 & & & \\
\midrule
      11 & & &  \\
\fi
\bottomrule
\end{tabular}
\end{center}
\end{problem}

\newpage

\begin{problem}{II}
  With your direct encodings of PHP($n$), run TBSAT in both direct and bucket mode to fill in the
  following table.
  The output line labeled
  ``\texttt{Total clauses}'' shows the number of proof clauses.
  What can you infer about how the proof size scales?  How do these compare to KISSAT?  How do these compare to each other?
\begin{center}
\ifshowsolutions
    \renewcommand{\arraystretch}{1.1}
\else
    \renewcommand{\arraystretch}{1.2}
\fi
    \begin{tabular}{rrrrr}
\toprule
\makebox[0.75in]{$n$} & \makebox[0.75in]{Input} & \makebox[0.75in]{Input}
& \makebox[1.5in]{Direct} & \makebox[1.5in]{Bucket} \\
\makebox[0.75in]{} & \makebox[0.75in]{Variables} & \makebox[0.75in]{Clauses}
& \makebox[1.5in]{Clauses} & \makebox[1.5in]{Clauses} \\
\midrule
\ifshowsolutions
      4 & 20  & 45 & 3,127 & 3,921 \\
      6 & 42 & 133 & 50,714 & 64,584  \\
      8 & 72 & 297 & 271,210 & 874,489 \\
      10 & 110 & 561 & 5,234,007 & 9,579,967 \\
      11 & 132 & 738 & 21,808,022 & 28,952,601 \\
\else
      4 & & & \\
\midrule
      6 & &  &  \\
\midrule
      8 & & &  \\
\midrule
      10 & & & \\
\midrule
      11 & & &  \\
\fi      
\bottomrule
\end{tabular}
\end{center}
\end{problem}

\begin{problem}{III}
The file
\pathname{generators}{cnf\_utilities.py} contains code
to generate encodings of various formulas as clauses.
Implement the linear at-most-one encoding scheme you devised in
Problem~\pref{sinz}.  To do so, fill in code for the functions
\texttt{lconEncode} and \texttt{rconEncode} in this file.
Within these functions, call
\texttt{writer.doClause} to generate the clauses.

Test your code by the following methods:
\begin{enumerate}
\item
Generate small instances of PHP with the `\texttt{-L}' and
`\texttt{-v}' flags set.  Convince yourself that the correct clauses are listed
\item
  Run these with KISSAT. Make sure they're unsatisfiable.
\item
  Try generating instances where the number of holes and the number of
  pigeons are the same.  These should be satisfiable.  Check the
  solutions generated by KISSAT and make sure they're valid.
\item
  On the satisfiable instances, try running TBSAT in mode where it
  generates multiple solutions (specified with the `\texttt{-m}'
  option.  Check that these solutions are all valid.
\end{enumerate}
\end{problem}

\newpage

\begin{problem}{III}
  Generate linear encodings of PHP($n$) and run KISSAT to fill in the
  following table.  To do so, you must give the name of the proof file
  on the command line.  The output line labeled
  ``\texttt{proof\_added}'' shows the number of proof clauses.
  How does the performance compare to the direct encoding?
\begin{center}
\ifshowsolutions
    \renewcommand{\arraystretch}{1.1}
\else
    \renewcommand{\arraystretch}{1.2}
\fi
    \begin{tabular}{rrrr}
\toprule
\makebox[0.75in]{$n$} & \makebox[0.75in]{Input} & \makebox[0.75in]{Input}
& \makebox[1.5in]{Proof} \\
\makebox[0.75in]{} & \makebox[0.75in]{Variables} & \makebox[0.75in]{Clauses}
& \makebox[1.5in]{Clauses} \\
\midrule
\ifshowsolutions
      4 & 32  & 45 & 50 \\
      6 & 72 & 103 & 342 \\
      8 & 128 & 185 & 1,717 \\
      10 & 200 & 291 & 11,933 \\
      11 & 242 & 353 & 30,539 \\
      12 & 242 & 353 & 4,108,200 \\
\else
      4 & & & \\
\midrule
      6 & &  &  \\
\midrule
      8 & & &  \\
\midrule
      10 & & & \\
\midrule
      11 & & &  \\
\midrule
      12 & & &  \\
\fi
\bottomrule
\end{tabular}
\end{center}
\end{problem}

\end{document}


